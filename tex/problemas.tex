\makeatletter
\def\input@path{{../}}
\makeatother
\documentclass[../main.tex]{subfiles}

\graphicspath{{ima/problemas}{../ima/problemas}}

% Aquí empieza el documento{{{
\begin{document}
\section{Problemas}%
\label{sec:problemas}


\thispagestyle{fancy}

\setcounter{subsection}{4}
\subsection{}%
Se tienen dos cilindros conductores de cobre, con igual longitud, cuyas
dimensiones se muestran en la siguiente figura.
Determina la resistencia del cilindro $(1)$ y $(2)$.
$L=20cm$ y $d=0.5cm$.

\begin{enumerate}[1)]
	\item Un cilindro de la $L$ de diámetro $d$, con un hueco en el centro
		de $\frac{d}{2}$ de diámetro
		\begin{align*}
			R &= \rho \frac{L}{A-A_h}\\
			R &= \rho \frac{0.2m}{A-A_h}\\
			R &= \rho \frac{0.2m}{ \frac{d^2}{4} - \frac{d^2}{16} }\\
			R &= \rho \frac{0.2m}{ \frac{3d^2}{16} }\\
			R &= 0.0175 *\cancel{10^{-6}}\Omega m \frac{3.2m*\cancel{10^6}}{3*25m^2}\\
			R &= 7.47 *10^{-4}\Omega
		\end{align*}
	\item Un cilindro de la $L$ de diámetro $d$.
		\begin{align*}
			R &= \rho \frac{L}{A}\\
			R &= \rho \frac{0.2m}{ \frac{d^2}{4}}\\
			R &= 0.0175 *\cancel{10^{-6}}\Omega m \frac{0.4m*\cancel{10^6}}{25m^2}\\
			R &= 2.8 *10^{-4}\Omega
		\end{align*}
\end{enumerate}

\setcounter{subsection}{6}
\subsection{}%

Calcula la resistencia eléctrica equivalente entre los bornes $A$ y $B$.
\begin{figure}[H]
	\centering
	\begin{tikzpicture}
	[
		circuit ee IEC,
		set resistor graphic=var resistor IEC graphic
	]
	\draw (0,0) to[resistor={info=$2\Omega$}] ++(0,2)
		to[resistor={info=$5\Omega$}] ++(2,0)
		to[resistor={info=$8\Omega$}] ++(0,-2)
		to[resistor={info=$1\Omega$}] ++(-2,0);

	\draw (2,2) to[resistor={info'=$1\Omega$}] ++(4,0)
		to[resistor={info'=$5\Omega$}] ++(-2,2)
		to[resistor={info=$3\Omega$}] ++(-2,-2)
		to[resistor={info=$12\Omega$}] ++(0,2)
		to[resistor={info=$2\Omega$}] ++(2,0);

	\draw (2,4) -- ++(0,1) node [contact] {} node[above] {$A$};
	\draw (2,0) -- ++(2,0) node [contact] {} node[right] {$B$};
	\end{tikzpicture}
\end{figure}

\begin{figure}[H]
	\centering
	\begin{tikzpicture}
	[
		circuit ee IEC,
		set resistor graphic=var resistor IEC graphic
	]
	\draw (0,0) to[resistor={info=$8\Omega$}] ++(0,2)
		-- ++(2,0)
		to[resistor={info=$8\Omega$}] ++(0,-2)
		-- ++(-2,0);

	\draw (2,2) -- ++(4,0)
		to[resistor={info'=$6\Omega$}] ++(-2,2)
		to[resistor={info=$3\Omega$}] ++(-2,-2)
		to[resistor={info=$12\Omega$}] ++(0,2)
		to[resistor={info=$2\Omega$}] ++(2,0);

	\draw (2,4) -- ++(0,1) node [contact] {} node[above] {$A$};
	\draw (2,0) -- ++(2,0) node [contact] {} node[right] {$B$};
	\end{tikzpicture}
\end{figure}

\begin{figure}[H]
	\centering
	\begin{tikzpicture}
	[
		circuit ee IEC,
		set resistor graphic=var resistor IEC graphic
	]
	\draw (2,0) to[resistor={info=$4\Omega$}] ++(0,2);

	\draw (2,2)
		to[resistor={info'=$2\Omega$}] ++(2,2)
		to[resistor={info'=$2\Omega$}] ++(-2,0)
		to[resistor={info'=$12\Omega$}] ++(0,-2);

	\draw (2,4) -- ++(0,1) node [contact] {} node[above] {$A$};
	\draw (2,0) -- ++(2,0) node [contact] {} node[right] {$B$};
	\end{tikzpicture}
\end{figure}

\begin{figure}[H]
	\centering
	\begin{tikzpicture}
	[
		circuit ee IEC,
		set resistor graphic=var resistor IEC graphic
	]
	\draw (2,0) to[resistor={info=$4\Omega$}] ++(0,2);

	\draw (2,2)
		to[resistor={info'=$12\Omega$}] ++(0,2)
		-- ++(2,0)
		to[resistor={info=$4\Omega$}] ++(0,-2)
		-- ++(-2,0);

	\draw (2,4) -- ++(0,1) node [contact] {} node[above] {$A$};
	\draw (2,0) -- ++(2,0) node [contact] {} node[right] {$B$};
	\end{tikzpicture}
\end{figure}

\begin{figure}[H]
	\centering
	\begin{tikzpicture}
	[
		circuit ee IEC,
		set resistor graphic=var resistor IEC graphic
	]
	\draw (2,0) to[resistor={info=$4\Omega$}] ++(0,2);

	\draw (2,2) to[resistor={info'=$3\Omega$}] ++(0,2);

	\draw (2,4) -- ++(0,1) node [contact] {} node[above] {$A$};
	\draw (2,0) -- ++(2,0) node [contact] {} node[right] {$B$};
	\end{tikzpicture}
\end{figure}

\begin{figure}[H]
	\centering
	\begin{tikzpicture}
	[
		circuit ee IEC,
		set resistor graphic=var resistor IEC graphic
	]
	\draw (2,0) to[resistor={info=$7\Omega$}] ++(0,4);

	\draw (2,4) -- ++(0,1) node [contact] {} node[above] {$A$};
	\draw (2,0) -- ++(2,0) node [contact] {} node[right] {$B$};
	\end{tikzpicture}
\end{figure}

\setcounter{subsection}{15}
\subsection{}%

A partir del circuito mostrado, determina $I$ si se sabe que $V_{AB}=50V$.

\begin{figure}[H]
	\centering
	\begin{tikzpicture}
	[
		circuit ee IEC,
		set resistor graphic=var resistor IEC graphic
	]
	\draw (0,0)++(0,4) to[battery={info'=$\mathcal{E}$}] (0,0);
	\draw (0,0) -- (6,0);

	\draw (6,0) to[resistor={info'=$5\Omega$}] (6,2);
	\draw (4,0) to[resistor={info'=$10\Omega$}] ++(0,2);
	\draw (2,0) to[resistor={info=$20\Omega$}] ++(0,2);
	\draw (2,2) to[resistor={info'=$10\Omega$}] ++(0,2);
	\draw (0,4) to[resistor={info=$R$, info'=$I$, direction info'={->}}] ++(2,0);

	\draw(6,2) -- ++(-4,0);

	\node [contact] at (2,4) {};
	\draw (2,4) node [above right] {$M$};
	\node [contact] at (6,2) {};
	\draw (6,2) node [above right] {$A$};
	\node [contact] at (6,0) {};
	\draw (6,0) node [below right] {$B$};
	\end{tikzpicture}
\end{figure}


\begin{align*}
	V_{AB} &= 50V\\
	\\
	R_{AB} &=
	\left(
		\frac{1}{20}
		+ \frac{1}{10}
		+ \frac{1}{5}
	\right)^{-1}
	\Omega\\
	R_{AB} &= \frac{20}{7} \Omega\\
	\\
	I &= I_{AB} = \frac{V_{AB}}{R_{AB}}\\
	I &= \frac{50V*7}{20\Omega}\\
	I &= \frac{35}{2} A
\end{align*}


\end{document}
%}}}
